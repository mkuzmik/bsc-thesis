\chapter{Wstęp}
\label{cha:wstep}

Rozwój technologii informacyjnych przyniósł duży postęp w obszarach przetwarzania oraz przechowywania informacji. Obecnie przychodzi nam się zmagać z ogromnymi ilościami danych, z których nie jesteśmy często w stanie wyciągnąć istotnych informacji. Co więcej, w dobie internetu dane są produkowane każdego dnia oraz na bieżąco magazynowane - wymaga to coraz bardziej zaawansowanych metod, które pozwolą nam je przetworzyć.

Duża część danych jest w formie tekstu - zapisana za pomocą języka naturalnego, zrozumiałym dla człowieka. Jest to nadal duże wyzwanie dla informatyki, która nie znalazła jednoznacznej odpowiedzi na komunikację maszyny z człowiekiem. Z tego powodu odnajdywanie wartościowych informacji zapisanych za pomocą języka naturalnego jest sporym wyzwaniem w dzisiejszym świecie. Pomimo braku idealnego rozwiązania istnieje wiele skutecznych technik, opartych na algorytmach nauczania maszynowego, które postaram się w tej pracy przybliżyć.

Celem pracy jest stworzenie systemu klasyfikującego dane tekstowe. Jego model opierał się będzie na podejściu data-driven \cite{wiki:datadriven} oraz danych ogólnodostępnych w internecie. Do implementacji systemu zostaną użyte klasyczne techniki uczenia maszynowego oraz algorytmy przetwarzania tekstu naturalnego. Aplikacja zostanie wdrożona w chmurze obliczeniowej z wykorzystaniem technologii serverless \cite{Serverless}.